\section{API Reference - Dynamic Components}

Unlike the static components discussed in Section 6, which operate on files,
SymtabAPI's dynamic components operate on a process. The dynamic components currently consist of the Dynamic Address Translation system, which translates between absolute addresses in a running process and static SymtabAPI objects. 
\subsection{Class AddressLookup}

The AddressLookup class provides a mapping interface for determining the address in a process where a SymtabAPI object is loaded. A single dynamic library may load at different addresses in different processes. The ‘address’ fields in a dynamic libraries’ symbol tables will contain offsets rather than absolute addresses. These offsets can be added to the libraries’ load address, which is computed at runtime, to determine the absolute address where a symbol is loaded. 

The AddressLookup class examines a process and finds its dynamic libraries and executables and each one’s load address. This information can be used to map between SymtabAPI objects and absolute addresses. Each AddressLookup instance is associated with one process. An AddressLookup object can be created to work with the currently running process or a different process on the same system.

On the Linux and Solaris platforms the AddressLookup class needs to read from the process’ address space to determine its shared objects and load addresses. By default, AddressLookup will attach to another process using a debugger interface to read the necessary information, or simply use memcpy if reading from the current process. The default behavior can be changed by implementing a new ProcessReader class and passing an instance of it to the createAddressLookup factor constructors. The ProcessReader class is discussed in more detail in Section 7.1.1.

When an AddressLookup object is created for a running process it takes a snapshot of the process’ currently loaded libraries and their load addresses. This snapshot is used to answer queries into the AddressLookup object, and is not automatically updated when the process loads or unloads libraries. The refresh function can be used to updated an AddressLookup object’s view of its process.

\begin{apient}
static AddressLookup *createAddressLookup(ProcessReader *reader = NULL)
\end{apient}
\apidesc{
This factory constructor creates a new AddressLookup object associated with the process that called this function. The returned AddressLookup object should be cleaned with the delete operator when it is no longer needed.
If the reader parameter is non-NULL on Linux or Solaris then the new AddressLookup object will use reader to read from the target process.
This function returns the new AddressLookup object on success and NULL on error.
}

\begin{apient}
static AddressLookup *createAddressLookup(PID pid, 
                                            ProcessReader *reader = NULL)
\end{apient}
\apidesc{
This factory constructor creates a new AddressLookup object associated with the process referred to by pid. The returned AddressLookup object should be cleaned with the delete operator when it is no longer needed.
If the reader parameter is non-NULL on Linux or Solaris then the new AddressLookup object will use reader to read from the target process.
This function returns the new AddressLookup object on success and NULL on error.
}

\begin{apient}
typedef struct {
    std::string name;
    Address codeAddr;
    Address dataAddr;
} LoadedLibrary;
\end{apient}

\begin{apient}
static AddressLookup *createAddressLookup(const std::vector<LoadedLibrary> &ll)
\end{apient}
\apidesc{
This factory constructor creates a new AddressLookup associated with a previously collected list of libraries from a process. The list of libraries can initially be collected with the getLoadAddresses function. The list can then be used with this function to re-create the AddressLookup object, even if the original process no longer exists. This can be useful for off-line address lookups, where only the load addresses are collected while the process exists and then all address translation is done after the process has terminated.
This function returns the new AddressLookup object on success and NULL on error.
}

\begin{apient}
bool getLoadAddresses(std::vector<LoadedLibrary> &ll)
\end{apient}
\apidesc{
This function returns a vector of LoadedLibrary objects that can be used by the
\texttt{createAddressLookup(const std::vector<LoadedLibrary> \&ll)} function to create a new AddressLookup object. This function is usually used as part of an off-line address lookup mechanism. 
This function returns true on success and false on error.
}

\begin{apient}
bool refresh()
\end{apient}
\apidesc{
When a AddressLookup object is initially created it takes a snapshot of the libraries currently loaded in a process, which is then used to answer queries into this API. As the process runs more libraries may be loaded and unloaded, and this snapshot may become out of date. 
An AddressLookup’s view of a process can be updated by calling this function, which causes it to examine the process for loaded and unloaded objects and update its data structures accordingly.
This function returns true on success and false on error.
}

\begin{apient}
bool getAddress(Symtab *tab, Symbol *sym, Address &addr)
\end{apient}
\apidesc{
Given a Symtab object, tab, and a symbol, sym, this function returns the address, addr, where the symbol can be found in the process associated with this AddressLookup. 
This function returns true if it was able to successfully lookup the address of sym and false otherwise.
}

\begin{apient}
bool getAddress(Symtab *tab, Offset off, Address &addr)
\end{apient}
\apidesc{
Given a Symtab object, tab, and an offset into that object, off, this function returns the address, addr, of that location in the process associated with this AddressLookup. 
This function returns true if it was able to successfully lookup the address of sym and false otherwise.
}

\begin{apient}
bool getSymbol(Address addr, Symbol * &sym, Symtab* &tab, bool close = false)
\end{apient}
\apidesc{
Given an address, addr, this function returns the Symtab object, tab, and Symbol, sym, that reside at that address. If the close parameter is true then getSymbol will return the nearest symbol that comes before addr; this can be useful when looking up the function that resides at an address.
This function returns true if it was able to find a symbol and false otherwise.
}

\begin{apient}
bool getOffset(Address addr, Symtab* &tab, Offset &off)
\end{apient}
\apidesc{
Given an address, addr, this function returns the Symtab object, tab, and an offset into tab, off, that reside at that address. 
This function returns true on success and false otherwise.
}

\begin{apient}
bool getAllSymtabs(std::vector<Symtab *> &tabs)
\end{apient}
\apidesc{
This function returns all Symtab objects that are contained in the process represented by this AddressLookup object. This will include the process’ executable and all shared objects loaded by this process. 
This function returns true on success and false otherwise.
}

\begin{apient}
bool getLoadAddress(Symtab *sym, Address &load_address)
\end{apient}
\apidesc{
Given a Symtab object, sym, that resides in the process associated with this AddressLookup, this function returns sym’s load address.
On the AIX system, where an object can have one load address for its code and one for its data, this function will return the code’s load address. Use getDataLoadAddress to get the data load address.
This function returns true on success and false otherwise.
}

\begin{apient}
bool getDataLoadAddress(Symtab *sym, Address &load_addr)
\end{apient}
\apidesc{
Given a Symtab object, sym, this function returns the load address of its data section. This function will return the data load address on AIX systems only, all other supported operating systems return zero.
This function returns true on success and false otherwise.}


\subsubsection{Class ProcessReader}
The implementation of the AddressLookup on Linux and Solaris requires it to be able to read from the target process’ address space. By default, reading from another process on the same system this is done through the OS’s debugger interface. A user can provide their own process reading mechanism by implementing a child of the ProcessReader class and passing it to the AddressLookup constructors. 
The API described in this section is an interface that a user can implement. With the exception of the ProcessReader constructor, these functions should not be called by user code.

The ProcessReader is defined, but not used, on non-Linux and Solaris systems.


\begin{apient}
PID pid
\end{apient}
\apidesc{
The pid member variable of ProcessReader refers to the PID of the process being read from.
}

\begin{apient}
ProcessReader(PID pid_)
\end{apient}
\apidesc{
This constructor for a ProcessReader should be called by any child class constructor.
}

\begin{apient}
virtual bool start() = 0
\end{apient}
\apidesc{
This function is called by AddressLookup before it begins a batch of reads from a process. Multiple readAddressSpace calls may follow the call to begin, which are then concluded with a call to done. This function may be useful for any initialization that needs to be done before reads. Note that this function signifies the start of a batch of reads, and multiple batches of reads occur during the lifetime of a ProcessReader. 
This function should return true on success and false on error.
}

\begin{apient}
virtual bool readAddressSpace(Address traced, unsigned amount, void *inSelf) = 0
\end{apient}
\apidesc{
This function should read amount bytes from the address at traced into the buffer pointed to by inSelf. 
This function should return true on success and false on error.
}

\begin{apient}
virtual bool done() = 0
\end{apient}
\apidesc{
This function is called to signify the completion of a batch of reads. It is guaranteed to be paired with a start call.
This function should return true on success and false on error.
}

\appendix

\section{Building SymtabAPI}

This appendix describes how to build SymtabAPI from source code, which can be
downloaded from http://www.paradyn.org or http://www.dyninst.org.  

\subsection{Building on Unix}

Building SymtabAPI on UNIX platforms is a four step process that involves:
unpacking the SymtabAPI source, installing any SymtabAPI dependencies,
configuring paths in make.config.local, and running the build.  

SymtabAPI's source code is packaged in a tar.gz format. If your SymtabAPI
source tarball is called \texttt{symtab\_src\_1.0.tar.gz}, then you could extract
it with the command \texttt{gunzip symtab\_src\_1.0.tar.gz; tar -xvf
symtab\_src\_1.0.tar}. This will create two directories: core and scripts.  

SymtabAPI has several dependencies, depending on what platform you are using,
which must be installed before SymtabAPI can be built. Note that for most of
these packages Symtab needs to be able to access the package's include files,
which means that development versions are required. If a version number is
listed for a packaged, then there are known bugs that may affect Symtab with
earlier versions of the package. 

\begin{center}
    \begin{tabular}{ll}
Linux/x86       & libdwarf-20060327 \\
                & libelf\\
                & libxml2\\
Linux/IA-64     & libdwarf-20060327 \\
                & libunwind-0.98.5 \\
                & lf \\
                & libxml2 \\
Linux/x86-64    & libdwarf-20060327 \\
                & libelf \\
                & libxml2 \\
Solaris/Sparc   & libxml2 \\
AIX/Power       & libxml2 \\
Windows/x86     & libxml2
\end{tabular}
\end{center}

At the time of this writing the Linux packages could be found at:
\begin{itemize}
    \item libdwarf - http://reality.sgiweb.org/davea/dwarf.html
    \item libelf - http://www.mr511.de/software/english.html
    \item libunwind - http://www.hpl.hp.com/research/linux/libunwind/download.php4
    \item libxml2 - http://xmlsoft.org/downloads.html
\end{itemize}

Once the dependencies for SymtabAPI have been installed, SymtabAPI must be
configured to know where to find these packages. This is done through
SymtabAPI's \texttt{core/make.config.local} file. This file must be written in
GNU Makefile syntax and must specify directories for each dependency.
Specifically, LIBDWARFDIR, LIBELFDIR and LIBXML2DIR variables must be set.
LIBDWARFDIR should be set to the absolute path of libdwarf library where
\texttt{dwarf.h} and \texttt{libdwarf.h} files reside. LIBELFDIR should be set
to the absolute path where \texttt{libelf.a} and \texttt{libelf.so} files are
located. Finally, LIBXML2DIR to the absolute path where libxml2 is located.

The next thing is to set DYNINST\_ROOT, PLATFORM, and LD\_LIBRARY\_PATH
environment variables. DYNINST\_ROOT should be set to the path of the directory
that contains core and scripts subdirectories.

PLATFORM should be set to one of the following values depending upon what operating system you are running on:
\begin{description}
    \item alpha-dec-osf5.1Tru64 UNIX on the Alpha processor
    \item i386-unknown-linux2.4Linux 2.4/2.6 on an Intel x86 processor
    \item ia64-unknown-linux2.4Linux 2.4/2.6 on an IA-64 processor
    \item rs6000-ibm-aix5.1AIX Version 5.1
    \item sparc-sun-solaris2.9Solaris 2.9 on a SPARC processor
    \item x86\_64-unknown-linux2.4Linux 2.4/2.6 on an AMD-64 processor
\end{description}

LD\_LIBRARY\_PATH variable should be set in a way that it includes libdwarf home
directory/lib and \${DYNINST\_ROOT}/\${PLATFORM}/lib directories.

Once \texttt{make.config.local} is set you are ready to build SymtabAPI. Change
to the core directory and execute the command make SymtabAPI. This will build
the SymtabAPI library. Successfully built binaries will be stored in a directory
named after your platform at the same level as the core directory. 

\subsection{Building on Windows}

SymtabAPI for Windows is built with Microsoft Visual Studio 2003 project and
solution files.   Building SymtabAPI for Windows is similar to UNIX in that it
is a four step process: unpack the SymtabAPI source code, install SymtabAPI's
package dependencies, configure Visual Studio to use the dependencies, and run
the build system. 

SymtabAPI's source code is distributed as part of a tar.gz package. Most popular
unzipping programs are capable of handling this format. Extracting the Symtab
tarball results in two directories: core and scripts. 

Symtab for Windows depends on Microsoft's Debugging Tools for Windows, which
could be found at http://www.microsoft.com/whdc/devtools/debugging/default.mspx
at the time of this writing. Download these tools and install them at an
appropriate location. Make sure to do a custom install and install the SDK,
which is not always installed by default. For the rest of this section, we will
assume that the Debugging Tools are installed at \texttt{C:\textbackslash
    Program Files\textbackslash Debugging
Tools for Windows}. If this is not the case, then adjust the following
instruction appropriately.

Once the Debugging Tools are installed, Visual Studio must be configured to use
them. We need to add the Debugging Tools include and library directories to
Visual Studios search paths. In Visual Studio 2003 select \texttt{Options...} from the
tools menu. Next select Projects and VC++ Directories from the pane on the left.
You should see a list of directories that are sorted into categories such as
`Executable files', `Include files', etc. The current category can
be changed with a drop down box in the upper right hand side of the Dialog. 

First, change to the `Library files' category, and add an entry that
points to \\
\texttt{C:\textbackslash Program Files\textbackslash Debugging Tools
for Windows\textbackslash sdk\textbackslash lib\textbackslash i386}. Make sure
that this entry is above Visual Studio's default search paths.

Next, Change to the `Include files' category and make a new entry in the
list that points to \texttt{C:\textbackslash Program Files\textbackslash Debugging Tools for
Windows\textbackslash sdk\textbackslash inc}. Also
make sure that this entry is above Visual Studio's default search paths. Some
users have had a problem where Visual Studio cannot find the \texttt{cvconst.h} file. You
may need to add the directory containing this file to the include search path.
We have seen it installed at \texttt{\$(VCInstallDir)\textbackslash
    ..\textbackslash Visual Studio SDKs\textbackslash DIA
SDK\textbackslash include}, although you may need to search for it. You also need to add the
libxml2 include path depending on the where the libxml2 is installed on the
system.  

Once you have installed and configured the Debugging Tools for Windows
you are ready to build Symtab. First, you need to create the directories where
Dyninst will install its completed build. From the core directory you need to
create the directories \texttt{..\textbackslash i386-unknown-nt4.0\textbackslash bin} and
\texttt{..\textbackslash i386-unknown-nt4.0\textbackslash lib}.
Next open the solution file core/SymtabAPI.sln with Visual Studio.   You can
then build SymtabAPI by select `Build Solution' from the build menu. This
will build the SymtabAPI library.

