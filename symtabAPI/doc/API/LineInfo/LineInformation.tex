\subsection{Class LineInformation}\label{LineInformation}
This class represents an entire line map for a module. This contains mappings from a line number within a source to the address ranges.

\begin{apient}
bool getAddressRanges (const char * lineSource, 
					   unsigned int LineNo, 
					   std::vector< AddressRange > & ranges )
\end{apient}
\apidesc{
This methos returns the address ranges in \code{ranges} corresponding to the line with line number \code{lineNo} in the source file \code{lineSource}. Searches within this line map.
Return \code{true} if at least one address range corresponding to the line number was found and returns \code{false} if none found.
}

\begin{apient}
bool getSourceLines( Offset addressInRange, std::vector< Statement *> & lines );
bool getSourceLines( Offset addressInRange, std::vector< LineNoTuple > & lines);

\end{apient}
\apidesc{
These methods returns the source file names and line numbers corresponding to the given address \code{addressInRange}. Searches within this line map. 
Return \code{true} if at least one tuple corresponding to the offset was found and returns \code{false} if none found.
}

\begin{apient}
bool addLine(const char * lineSource, unsigned int lineNo, 
            unsigned int lineOffset, Offset lowInclusiveAddr, 
            Offset highExclusiveAddr)
\end{apient}
\apidesc{
This method adds a new line to the line Map. \code{lineSource} represents the source file name. \code{lineNo} represents the line number.
}

\begin{apient}
bool addAddressRange(Offset lowInclusiveAddr, Offset highExclusiveAddr,
                    const char* lineSource, unsigned int lineNo, 
                    unsigned int lineOffset = 0);
\end{apient}
\apidesc{
This method adds an address range \code{[lowInclusiveAddr, highExclusiveAddr)} for the line with line number \code{lineNo} in source file \code{lineSource}. 
}

\begin{apient}
LineInformation::const_iterator begin() const
\end{apient}
\apidesc{
This method returns an iterator pointing to the beginning of the line information for the module.
This is useful for iterating over the entire line information present in a module. An example described in Section 6.3.3 gives more information on how to use \code{begin()} for iterating over the line information.
}

\begin{apient}
LineInformation::const_iterator end() const
\end{apient}
\apidesc{
This method returns an iterator pointing to the end of the line information for the module.
This is useful for iterating over the entire line information present in a module. An example described in Section 6.3.3 gives more information on how to use \code{end()} for iterating over the line information.
}
