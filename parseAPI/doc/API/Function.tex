\subsection{Class Function}

\definedin{CFG.h}

The Function class represents the protion of the program CFG that is reachable through intraprocedural control flow transfers from the function's entry block. Functions in the ParseAPI have only a single entry point; multiple-entry functions such as those found in Fortran programs are represented as several functions that ``share'' a subset of the CFG. Functions may be non-contiguous and may share blocks with other functions. 

\begin{center}
\begin{tabular}{ll}
\toprule
FuncSource & Meaning \\
\midrule
RT & recursive traversal (default) \\
HINT & specified in CodeSource hints \\
GAP & speculative parsing heuristics \\
GAPRT & recursive traversal from speculative parse \\
ONDEMAND & dynamically discovered at runtime \\
\bottomrule
\end{tabular}
\end{center}

\apidesc{Return type of function \code{src()}; see description below.}

\begin{center}
\begin{tabular}{ll}
\toprule
FuncReturnStatus & Meaning \\
\midrule
UNSET & unparsed function (default) \\
NORETURN & will not return \\
UNKNOWN & cannot be determined statically \\
RETURN & may return \\
\bottomrule
\end{tabular}
\end{center}

\apidesc{Return type of function \code{retstatus()}; see description below.}

\begin{apient}
typedef std::vector<Block*> blocklist
typedef std::set<Edge*> edgelist
\end{apient}
\apidesc{Containers for block and edge access. Library users \emph{must not} rely on the underlying container type of std::set/std::vector lists, as it is subject to change.}

\begin{tabular}{p{1.25in}p{1.125in}p{3.125in}}
	Method name & Return type & Method description \\
	\hline
	name & string & Name of the function. \\
	addr & Address & Entry address of the function.  \\
	entry & Block * & Entry block of the function. \\
	parsed & bool & Whether the function has been parsed. \\
	blocks & blocklist \& & List of blocks contained by this function sorted by entry address. \\
	callEdges & edgelist \& & List of outgoing call edges from this function. \\
	returnBlocks & blocklist \& & List of all blocks ending in return edges. \\
	exitBlocks & blocklist \& & List of all blocks that end the function, including blocks with no out-edges. \\
	hasNoStackFrame & bool & True if the function does not create a stack frame. \\
	savesFramePointer & bool & True if the function saves a frame pointer (e.g. \%ebp). \\
	cleansOwnStack & bool & True if the function tears down stack-passed arguments upon return. \\
	region & CodeRegion * & Code region that contains the function. \\
	isrc & InstructionSource * & The InstructionSource for this function. \\
	obj & CodeObject * & CodeObject that contains this function. \\
	src & FuncSrc & The type of hint that identified this function's entry point. \\
	restatus & FuncReturnStatus * & Returns the best-effort determination of whether this function may return or not. Return status cannot always be statically determined, and at most can guarantee that a function \emph{may} return, not that it \emph{will} return. \\
	getReturnType & Type * & Type representing the return type of the function. \\
\end{tabular}

\begin{apient}
Function(Address addr, string name, CodeObject * obj, CodeRegion * region, InstructionSource * isource);
\end{apient}
\apidesc{Creates a function at \code{addr} in the code region specified. 
Insructions for this function are given in \code{isource}.}

\begin{apient}
std::vector<FuncExtent *> const& extents()
\end{apient}
\apidesc{Returns a list of contiguous extents of binary code within the function.}

\begin{apient}
void setEntryBlock(block * new_entry
\end{apient}
\apidesc{Set the entry block for this function to \code{new\_entry}.}

\begin{apient}
void set_retstatus(FuncReturnStatus rs)
\end{apient}
\apidesc{Set the return status for the function to \code{rs}}

\begin{apient}
void removeBlock( Block* )
\end{apient}
\apidesc{Remove a basic block from the function}
