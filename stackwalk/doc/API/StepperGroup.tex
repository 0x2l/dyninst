\subsubsection{Class StepperGroup}
\label{subsec:steppergroup}
\definedin{steppergroup.h}

The \code{StepperGroup} class contains a collection of \code{FrameStepper} objects. The
\code{StepperGroup}'s primary job is to decide which \code{FrameStepper} should be used to
walk through a stack frame given a return address. The default \code{StepperGroup}
keeps a set of address ranges for each \code{FrameStepper}. If multiple \code{FrameStepper}
objects overlap an address, then the default \code{StepperGroup} will use a priority
system to decide.

\code{StepperGroup} provides both an interface and a default implementation of that
interface. Users who want to customize the \code{StepperGroup} should inherit from this
class and re-implement any of the below virtual functions.

\begin{apient}
virtual StepperGroup(Walker *walker)
\end{apient}
\apidesc{
    This factory constructor creates a new \code{StepperGroup} object associated
with \code{walker}. 
}

\begin{apient}
virtual bool addStepper(FrameStepper *stepper)
\end{apient}
\apidesc{
    This method adds a new \code{FrameStepper} to this \code{StepperGroup}. The
    newly added stepper will be tracked by this \code{StepperGroup}, and it will
    be considered for use when walking through stack frames. 

    This method returns \code{\code{true}} if it successfully added the
    \code{FrameStepper}, and \code{\code{false}} on error.
}

\begin{apient}
virtual bool addAddressRanges(
    const std::vector<std::pair<Address, Address> >&ranges,
    const FrameStepper *stepper) = 0
\end{apient}
\apidesc{
    This method associates a set of address ranges, \code{ranges}, with a \code{FrameStepper},
    stepper. These address ranges contain objects in the process' code space
    that create stack frames that \code{stepper} can walk through.

    The default \code{StepperGroup} will use \code{stepper} to walk through a Frame object (by
    returning it from findStepperForAddr) if the Frame object's RA falls within
    a range registered by this method. A Frame object, frame, falls within a
    range, range[i], if . If multiple \code{FrameStepper} objects have overlapping
    ranges, then the default \code{StepperGroup} will use the one with the highest
    priority first (see \code{FrameStepper}::getPriority in Section~
    \ref{subsec:framestepper}).

    For example, suppose this \code{FrameStepper} was designed to walk through a signal
    handler frame on Linux/x86. During initialization the \code{FrameStepper} inspects
    the target process' vsyscall page and finds that signal handlers will appear
    on the call stack with a RA between \code{0xffffe000} and \code{0xffffe400}. It then
    registers this range with its \code{StepperGroup} using addAddressRanges. If the
    \code{StepperGroup} encounters an RA in this range, it then uses the signal handler
    \code{FrameStepper} to walk through it.

    Suppose another \code{FrameStepper} was designed to walk through regular stack
    frames created by ABI-compliant functions. This \code{FrameStepper} will be used as
    a general catch-all if no other \code{FrameStepper} can walk through a Frame
    object. The \code{FrameStepper} can register itself with an address range that
    spans the whole address space, and a lower priority than the signal handler
    \code{FrameStepper}. The \code{StepperGroup} will then use the signal handler \code{FrameStepper}
    to step through signal handlers, and this \code{FrameStepper} to step through any
    other Frame object.

    This method returns \code{true} on success and \code{false} if there is an
    error.
}

\begin{apient}
virtual bool removeAddressRanges(
    const std::vector<std::pair<Address, Address > > &ranges, 
    const FrameStepper *stepper) = 0
\end{apient}
\apidesc{
    This method removes a FrameStepper's address range from a StepperGroup. See
    addAddressRange for more details on how StepperGroup and FrameStepper
    objects use address ranges. The address ranges specified by ranges will be
    deleted from stepper's address ranges. For example, if the address range
    \code{0x1000} to \code{0x2000} was registered to a FrameStepper named \code{foo}, and then
    removeAddressRanges was used to remove the address range \code{0x1500} to
    \code{0x1600}
    out of \code{foo}, then \code{foo} would have two address ranges associated with it:
    \code{0x1000} to \code{0x1500} and \code{0x1600} to \code{0x2000}.

	This function returns \code{true} on success and \code{false} on error.
}

\begin{apient}
virtual bool findStepperForAddr(Address addr, FrameStepper* &out, 
                                const FrameStepper *last_tried = NULL)
\end{apient}
\apidesc{
    Given an address that points into a function (or function-like object),
    addr, this method decides which \code{FrameStepper} should be used to walk through
    the stack frame created by the function at that address. A pointer to the
    \code{FrameStepper} will be returned in parameter \code{out}. 

    It may be possible that the \code{FrameStepper} this method decides on is unable to
    walk through the stack frame (it returns \code{gcf\_not\_me} from
    \code{FrameStepper::getCallerFrame}). In this case StackwalkerAPI will call
    findStepperForAddr again with the last\_tried parameter set to the failed
    \code{FrameStepper}. findStepperForAddr should then find another \code{FrameStepper} to
    use. Parameter \code{last\_tried} will be set to NULL the first time getStepperToUse
    is called for a stack frame.

    The default version of this method uses address ranges to decide which
    \code{FrameStepper} to use. The address ranges are contained within the process'
    code space, and map a piece of the code space to a \code{FrameStepper} that can
    walk through stack frames created in that code range. If multiple
    \code{FrameStepper} objects share the same range, then the one with the highest
    priority will be tried first.
	
    This method returns \code{true} on success and \code{false} on failure.  
}

\begin{apient}
Walker *getWalker() const
\end{apient}
\apidesc{
	This method returns the Walker object that associated with this StepperGroup.
}
