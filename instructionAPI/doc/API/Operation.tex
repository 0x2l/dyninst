\subsection{Operation Class}
\label{sec:operation}

An Operation object represents a family of opcodes (operation encodings) that
perform the same task (e.g. the \code{MOV} family). It includes information
about the number of operands, their read/write semantics, the implicit register
reads and writes, and the control flow behavior of a particular assembly
language operation. It additionally provides access to the assembly mnemonic,
which allows any semantic details that are not encoded in the Instruction
representation to be added by higher layers of analysis.

As an example, the \code{CMP} operation on IA32/AMD64 processors has the following properties:
\begin{itemize}
\item Operand 1 is read, but not written
\item Operand 2 is read, but not written
\item The following flags are written:
\begin{itemize}
\item Overflow
\item Sign
\item Zero
\item Parity
\item Carry
\item Auxiliary
\end{itemize}
\item No other registers are read, and no implicit memory operations are performed
\end{itemize}

Operations are constructed by the \code{InstructionDecoder} as part of the
process of constructing an Instruction. 

\begin{apient}
const Operation::registerSet & implicitReads () const
\end{apient}
\apidesc{
Returns the set of registers implicitly read (i.e. those not included in the
operands, but read
anyway).
}

\begin{apient}
const Operation::registerSet & implicitWrites () const
\end{apient}
\apidesc{
Returns the set of registers implicitly written (i.e. those not included in the
operands, but written
anyway).
}

\begin{apient}
    std::string format() const
\end{apient}
\apidesc{
    Returns the mnemonic for the operation. Like \code{instruction::format}, this is
exposed for debug-
ging and will be replaced with stream operators in the public interface.
}

\begin{apient}

\end{apient}
\apidesc{

}

\begin{apient}

\end{apient}
\apidesc{

}

\begin{apient}

\end{apient}
\apidesc{

}

\begin{apient}

\end{apient}
\apidesc{

}

\begin{apient}

\end{apient}
\apidesc{

}

\begin{apient}

\end{apient}
\apidesc{

}

