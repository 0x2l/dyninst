\section{Examples} \label{sec-example}
To illustrate the ideas of PatchAPI, we present some simple code examples that
demonstrate how the API can be used.

\subsection{Using the public interface}
The basic flow of doing code patching is to first find some points in a program,
and then to insert, delete or update a piece of code at these points.
\subsubsection{Point Finding} \label{sec-example-pt}

\lstset{language=[GNU]C++,basicstyle=\fontfamily{fvm}\selectfont\small}
\lstset{numbers=left}
\begin{lstlisting}
PatchFunction *func = ...;
PatchBlock *block = ...;
PatchEdge *edge = ...;

PatchMgr *mgr = ...;

std::vector<Point*> pts;
mgr->findPoints(Scope(func),
                Point::FuncEntry | 
                Point::PreCall | 
                Point::FuncExit,
                back_inserter(pts));
mgr->findPoints(Scope(block),
                Point::BlockEntry,
                back_inserter(pts));
mgr->findPoints(Scope(edge),
                Point::EdgeDuring,
                back_inserter(pts));
\end{lstlisting}
The above code shows how to use the PatchMgr::findPoints method to find some
instrumentation points. There are three invocations of findPoints. For the first
invocation (Line 8), it finds points only within a specific function
\emph{func}, and output the found points to a vector \emph{pts}. The result
should include all points at this function's entry, before all function calls
inside this function, and at the function's exit. Similarly, for the second
invocation (Line 13), it finds points only within a specific basic \emph{block},
and the result should include the point at the block entry. Finally, for the
third invocation (Line 16), it finds the point at a specific CFG \emph{edge}
that connects two basic blocks.

\subsubsection{Code Patching}

\lstset{language=[GNU]C++,basicstyle=\fontfamily{fvm}\selectfont\small}
\lstset{numbers=left}
\begin{lstlisting}
MySnippet snip;
Snippet<MySnippet>::ptr snippet = Snippet<MySnippet>::create(snip);

Patcher patcher(mgr);
for (vector<Point*>::iterator iter = pts.begin();
     iter != pts.end(); ++iter) {
  Point* pt = *iter;
  patcher.add(PushBackCommand::create(pt, snippet));
}
patcher.commit();
\end{lstlisting}

The above code is to insert the same code \emph{snippet} to all points
\emph{pts} found in Section~\ref{sec-example-pt}. We'll explain the snippet
(Line 1 and 2) in the example in Section~\ref{sec-example-snip}. Each point
maintains a list of snippet instances, and the PushBackCommand is to push a
snippet instance to the end of that list. An instance of Patcher is to represent
a transaction of code patching. In this example, all snippet insertions (or all
PushBackCommands) are performed atomically when the Patcher::commit method is
invoked. That is, all snippet insertions would succeed or all would fail.

\subsection{Using the plugin interface}

\subsubsection{Address Space}
\lstset{language=[GNU]C++,basicstyle=\fontfamily{fvm}\selectfont\small}
\lstset{numbers=left}
\begin{lstlisting}
class MyAddrSpace : public AddrSpace {
  public:
    ...
    virtual Address malloc(PatchObject* obj, size_t size, Address near) {
      Address buffer = ...
      // do memory allocation here
      return buffer;
    }
    virtual bool write(PatchObject* obj, Address to_addr, Address from_addr,
                       size_t size) {
      // copy data from the address from_addr to the address to_addr
      return true;
    }
    ...
};
\end{lstlisting}
The above code is to implement the address space plugin, in which, a set of
memory management methods should be specified, including malloc, free, realloc,
write and so forth. The instrumentation engine will utilize these memory
management methods during the code patching process. For example, the
instrumentation engine needs to \emph{malloc} a buffer in Mutatee's address
space, and then \emph{write} the code snippet into this buffer.

\subsubsection{Snippet Representation} \label{sec-example-snip}
\lstset{language=[GNU]C++,basicstyle=\fontfamily{fvm}\selectfont\small}
\lstset{numbers=left}
\begin{lstlisting}
struct MySnippet {
  void* binary_blob;
};
MySnippet snip;
Snippet<MySnippet>::ptr snippet = Snippet<MySnippet>::create(snip);
\end{lstlisting}
The above code illustrates how to plug in a user-defined snippet representation
\emph{MySnippet}, which simply uses a binary blob as the code snippet for
insertion. The snippet representation \emph{MySnippet} is specified as a
template parameter to PatchAPI's \emph{Snippet} class, so that the
instrumentation engine can recognize \emph{MySnippet} in the later code
generation phase.

\subsubsection{Code Parsing}
\lstset{language=[GNU]C++,basicstyle=\fontfamily{fvm}\selectfont\small}
\lstset{numbers=left}
\begin{lstlisting}
class MyFunction : public PatchFunction {
  ...
};
class MyCFGMaker : public CFGMaker {
  public:
    ...
    virtual PatchFunction* makeFunction(ParseAPI::Function *f, PatchObject* o) {
      return new MyFunction(f, o);
    }
    ...
};
\end{lstlisting}
Programmers can augment PatchAPI's CFG structures by annotating their own data.
In this case, a factory class should be built by inheriting from the
CFGMaker class, to create the augmented CFG structures. The factory class will
be used for CFG parsing.

\subsubsection{Instrumentation Engine}
\lstset{language=[GNU]C++,basicstyle=\fontfamily{fvm}\selectfont\small}
\lstset{numbers=left}
\begin{lstlisting}
class MyInstrumenter : public Instrumenter {
  public:
    virtual bool run() {
      // Specify how to install instrumentation
    }
};
\end{lstlisting}
Programmers can customize the instrumentation engine by extending the
Instrumenter class, and implement the installation of instrumentation inside the
method \emph{run()}.

\subsubsection{Plugin Registration}
\lstset{language=[GNU]C++,basicstyle=\fontfamily{fvm}\selectfont\small}
\lstset{numbers=left}
\begin{lstlisting}
MyCFGMakerPtr cm = ...
PatchObject* obj = PatchObject::create(..., cm);

MyAddrSpacePtr as = ...
as->loadObject(obj);

MyInstrumenter inst = ...
PatchMgrPtr mgr = PatchMgr::create(as, ..., inst);

MySnippet* snip = ...
Snippet<MySnippet>::ptr snippet = Snippet<MySnippet>::create(snip);
\end{lstlisting}
The above code shows how to register the above four types of plugins.  An
instance of the factory class for creating CFG structures is registered to an
PatchObject (Line 1 and 2), which is in turn loaded into an instance of
AddrSpace (Line 4 and 5). The AddrSpace (or its subclass implemented by
programmers) instance is passed to PatchMgr::create (Line 7 and 8), together
with an instance of Instrumenter (or its subclass). Finally, the user-specified
snippet representation is passed as a template parameter to the Snippet
class (Line 10 and 11). Therefore, all plugins are glued together in PatchAPI.
